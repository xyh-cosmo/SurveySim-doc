\documentclass[a4paper,11pt]{ctexrep}
\usepackage{xcolor}
\usepackage{graphicx}
\usepackage{amsmath}
\usepackage{setspace}
\usepackage{bookmark}
\usepackage{textcomp} %提供\textdegree
\usepackage[text={6.5in,9in},centering]{geometry}
\usepackage{array}
\usepackage{multirow}
\usepackage{ulem}
\usepackage{bm}
\usepackage{booktabs} % referred from <Latex_Cookbook>

\ctexset{today=big}


% 一些用于给字体增加颜色的命令
\newcommand{\RT}[1]{\textcolor{red}{#1}}
\newcommand{\BT}[1]{\textcolor{blue}{#1}}
\newcommand{\GT}[1]{\textcolor{green}{#1}}
\newcommand{\MT}[1]{\textcolor{magenta}{#1}}

% 在MacOS下可以使用\heiti,\kaishu来设置字体,使用方法如下
% {\heiti 中文内容}

\begin{document}

\title{\bf CSS-OS——运行编排仿真}
\author{\heiti 许优华}
\maketitle

\abstract{
\kaishu 本文档是关于CSS-OS——运行编排仿真程序的相关说明。本文档的主要目的是记录运行编排仿真中所考虑的各项
影响观测的因素(包括设施的硬件约束、天文观测条件),模拟程序的设计,各个模块功能的介绍和测试,以及相关的
工作日志。}

\tableofcontents

\input{chapters/survey}
\input{chapters/constraints}
% !TEX root = ../CSS-OS.tex

\chapter{巡天策略}

具体的巡天策略可以看作是“巡天模拟”的参数输入。不同“巡天策略”的模拟所需要的计算资源是不同的,因此
这些方面需要进行非常详细和周密的考虑。

\BT{3月16日的讨论涉及到了策略的优化。讨论过程中提及到了多种策略方案,但目前还没有办法进行实现。。。}

目前对巡天策略已经进行过比较详细的分析,已经在仿真程序中进行了多处、多次改动,仿真结果也确实得到了
显著的改进。

\input{chapters/code}
\input{chapters/result}
% !TEX root = ../CSS-OS.tex

\chapter{模拟结果的检验}

{\heiti 检验的工作与唐怀金合作。}


\section{检验模拟结果的基本思路及各项判断条件的具体定义}


\subsection{卫星受到光照(阳照区)的判断}
\RT{此处内容由唐怀金提供}

模型一:地球中心$O$,与卫星$E$和太阳$S$。$R$是地球半径.太阳角半径16角分,地平线上蒙气差为35.4角分。
卫星受到光照的判断条件是 
\begin{equation}
\theta_0 \le \theta_1+ \theta_2+ \theta_3
\end{equation}
其中
\begin{eqnarray}
\theta_0 &=& \arccos \frac{\bm{OS} \cdot \bm{OE}}{|\bm{OS}| \cdot |\bm{OE}|},\\
\theta_1 &=& \arccos{\frac{R}{|\bm{OS}|}},\\
\theta_2 &=& \arccos{\frac{R}{|\bm{OE}|}},\\
\theta_3 &=& (16+35.4\times 2)/60.
\end{eqnarray}
 
模型二:卫星未受到光照的判断条件是地球中心到太阳与卫星之间的连线的距离$h$满足如下关系
\begin{eqnarray}
h < R \cos\left( \frac{16+35.4\times2}{60} \right),
\end{eqnarray}
且$\theta_0$为钝角,此模型为最早使用的模型。 其中 
\begin{eqnarray}
h=\frac{\bm{OS} \cdot \bm{OE}}{|\bm{SE}|}
\end{eqnarray}

\subsection{杂散光模型(唐怀金提供)}
原点位于地球中心$O$,目标点P的坐标为$(P_x,P_y,P_z)$,卫星的位置STL为$(STL_x,STL_y,STL_z)$,
太阳位置SUN为$(SUN_x,SUN_y,SUN_z)$,地球半径为$R$。天顶距$ANGLE_z$,暗边角阈值$ANGLE_{dark}$,
亮边角阈值$ANGLE_{light}$.

遮挡角:
\begin{eqnarray}
ANGLE_1 = \arcsin\left(\frac{R}{L_{OP}}\right),
\end{eqnarray}
简便起见,这里不严格区分弧度与角度的差别。

卫星在地球上的视线切线是否与晨昏线相交分为三种情况:
\begin{itemize}

\item [一:]与晨昏线不相交且卫星位于阴影区域,计算观测目标点的天顶距,判断是否为暗边角的判断条件:
\begin{eqnarray}
ANGLE_z \le 180-ANGLE_1-ANGLE_{dark}
\end{eqnarray}

\item [二:]与晨昏线不相交且卫星位于阳照区,计算观测目标点的天顶距,判断是否为亮边角的判断条件:
\begin{eqnarray}
ANGLE_z \le 180-ANGLE_1-ANGLE_{light}
\end{eqnarray}

\item [三:]与晨昏线相交。\\
第一步:以亮边角的天顶距做初始判断,满足条件则通过:
\begin{eqnarray}
ANGLE_z \le 180-ANGLE_1-ANGLE_{light};
\end{eqnarray}

第二步:不满足第一步条件,但是满足以下两个条件:
\begin{eqnarray}
ANGLE_z &\ge& 180-ANGLE_1-ANGLE_{light}\\
ANGLE_z &\le& 180-ANGLE_1-ANGLE_{dark}
\end{eqnarray}
则计算卫星、地心和目标点共面的切点$P_{Q}$(可由平面几何推导),切点必须位于阴影区。
若能计算出晨昏线与视线切线环的焦点,则考虑阳照区的影响。

\end{itemize}

\section{目前的检验结果}
\begin{itemize}
\item 阳照区\\
目前对各项限制条件的检验结果中“阳照区”的判断有$4.5\%$的不符合率,是目前所有限制条件中
最严重的情况。目前已经将这些不符合的模拟结果调取出来进行细致的检查。\RT{目前猜测,出现
这么高的不符合率应该是由张鑫的程序与唐怀金的检验程序之间的一些假设之间的差异所导致的:
张鑫的程序对阳照区的判断基于“照射到地球的太阳光为平行光”这样一个假设,相比较之下,唐怀金
的检验程序在这一块的处理更加接近实际情况。}

\item SAA区域\\

\item 太阳能帆板\\


\item 地球杂散光、地气光\\


%\item 暗边、亮边夹角\\


\item 太阳夹角\\


\item 月球夹角\\

\end{itemize}
\section{检验中出现的不符合情况及原因排查}


\section{检验各项条件的具体思路}
\RT{\heiti 这个是一个全新的章节,目的是详细讨论检验各种限制的具体做法,具体如何实现,甚至
包括如何检验这些测试本身(这就需要构造出一些具体的、知道结果的例子来来进行验证)。这一
章节重点围绕动态限制条件,也就是CMG和能源这两方面;其余的条件都是静态的,或者准确点说是
前后没有什么因果联系的,比如各种遮挡角度的计算。}

\subsection{CMG}
昨天(Wed Jul-10, 2018)在生成相机运行序列的时候,顺便检验了一下连续两次曝光之间,由于
机动原因导致的指向随时间的变化。

检验的具体思路如下:知道了两个天区的指向,同时也知道有一个旋转矩阵,可以描述这个转动过程,
并保证不产生像旋(这个矩阵的公式可以方便地推导出来)。假设这两个天区的指向分别为
$\bm{P}_{init}$和$\bm{P}_{final}$,相应的旋转矩阵记为
$\bm{R}(t,t_{init};\bm{P}_{init},\bm{P}_{final})$,那么旋转过程的任意时刻的指向可以写作:
\begin{eqnarray}
\bm{P}(t) = \bm{R}(t,t_{init}) \cdot \bm{P}_{init} \text{,}
\end{eqnarray}
其中$t_{init} \le t \le t_{final}$;此处为了简便,旋转矩阵中忽略掉初始和末尾指向。

整个拍摄周期划分为多个“状态”,每一个“状态”又是由各个“组部件的状态”集合而成的;这些组部件的工作
状态在当前的“状态”下维持不变。具体检验过程中我是这么来做的:在转动到下一个指向之前的那个状态
下,根据巡天编排模拟程序输出结果中的前后两个指向来确定出旋转轴,然后假设旋转是匀速的,那样
一来就可以直接计算出旋转过程中任意时刻的指向了。当计算出末尾指向后与模拟结果给出的指向进行
比较,发现存在很大的差别。

为了找出原因,仔细检查了旋转矩阵的计算等等,确认无误。然后就是检查巡天编排的模拟程序,找到
计算CMG旋转角度的那些代码,然后一眼就看到在将天区指向传递给计算旋转角度的函数时,忘记做
角度的单位转换了。当时还根据天文习惯和传统数学对球坐标系的不同使用习惯,将天顶角($\theta$)
进行了变换,但就是忘记把黄经(ra)和黄纬(dec)的单位给转换到弧度了。\RT{(重大失误!
这样一来,不仅CMG的转动角以及CMG的状态受到影响,连转动时间也受到影响了)}

\subsection{Solar Energy}

\input{chapters/check_log}
% !TEX root = ../CSS-OS.tex

\chapter{模拟程序的优化}

\section{7月}

\RT{\heiti 前阵子发现了程序中的一个严重bug,就是在计算CMG的转动角度的时候,传递的天区位置没有正确转换
到弧度单位下的数值,因而所给出的序列全部都错了!修正了bug之后重新运行了几组,发现不能在10年内完成既定
的巡天任务。}

\subsection{7月18日}

\BT{\heiti 在上个周末,苦苦思索、分析模拟运行的中间状态后,意识到张鑫原先设计的策略中有这样一条,就是
在当前搜索到的可观测天区内,如果有“极深场”,那么就优先观测“极深场”,这导致了很多不必要的大角度转动。
优先观测“极深场”引起的这些不必要的大角度转动造成了很多“可观测时间”的浪费,最终后果就是减少了10年里
真正可以用于曝光的时间。}

\GT{\heiti 由此可以提炼出一条优化规划的法则:“任何时候都要减少转动的时间”!}

从7月17日傍晚6点半至18日半夜,已经完成4组取消了“极深场优先策略”的模拟,虽然离预期的目标还差一些,
但是相比之下时间的利用率有了很大的改进,最终未能被观测到的天区数目减少了很多。尽管已经能保证17500
平方度的大面积巡天,但是“极深场”的面积离目标还差100多平方度。

将“天区覆盖率”随时间的关系统计出来之后,发现从8.5年开始,覆盖速率有较为明显的下降。下降的主要原因
是低黄纬、低银纬的天区很快就被覆盖完了,剩下的未观测区域相互之间离得比较远,因此望远镜的指向在从
一个区域转动到另外一个区域的时候就需要更多的时间,进而造成了时间上的不必要的浪费。

18日晚又提交了4组模拟,这一次对高纬度的区域进行了调整,提高了高纬度的阈值,尽可能使得高纬度的区域
尽快被观测完,延缓低纬度区域被观测完的时间,这样做的目的是为了在10年内能够让“天区覆盖率”维持在一个
平均水平,不至于在后期的时候速率降到很低的值。\RT{这4组模拟所采用的天区划分完全一致,只是对“高纬度”
所选取的阈值不同;另外将“连续观测优先”的策略延迟到第8年才开始。}

\subsection{7月19日}
今天想到,针对昨天发现的“极深场”面积不达标的问题,应该可以比较简单地解决。在根据CMG转动角度进行
权重修改的代码部分,在角度很小的两种情况下(小于5度和10度的两种情况下,此时转动所需的时间较少),
可以加一个判断语句,如果当前“预选”天区属于“极深场”,那么就进一步优化权重,这样一来也可以保证不需要
大角度的转动了。下午3点40分的时候已经提交了4组这样的模拟,明早看结果。

仅仅使用“天区覆盖率”还不能够比较准确地来度量巡天效率,应该选择一段时间内(比如60天)的总曝光时间
所占的比例。目前程序中还不能实时地计算这个数值,但根据所生成的序列还是可以进行统计的。(需要写一
个脚本来进行统计)

\RT{因为“极深场”区域的每一个天区都需要重复观测4次,那么在满足一定条件的时候,就可以省去转动、直接
开始多次曝光了。但是要进行这样的判断需要提前知道轨道的情况,感觉会比较的复杂。。。}

\MT{突然又意识到,其实使用“连续观测”的策略进行权重修改会干扰后面根据CMG转动角大小进行权重分配,
于是干脆将“连续观测”这个策略取消,现在已经再次提交了四组模拟。(张鑫原先实现的“连续观测”策略可能会
导致大角度的转动???)}

\subsection{7月20日}

还有一些小技巧可以进一步减少时间的浪费。一个具体的例子就是在当前比较容易观测的区域属于的时候,可以
连续地进行曝光,(几乎)不需要进行转动,从而可以节省下70秒的时间(根据XXX提供的数据,角度小于
1度的转动需要的时间为70秒)。要使得“极深场”面积达到400平方度,总的曝光次数至少为43636次,可以
利用来节约时间的曝光次数为32727次。如果其中有一半可以用来不转动、节约时间,那么一共可以节省下约
13天的时间,这些时间又几乎都是可以进行正常观测的,因此转化为大面积巡天曝光次数,大约为4580次
(具体估算为400/1.1*30*1.5*70/(150+100)=4581)。然而,目前实际给“极深场”规划的天区面积要多出很多,
比实际需要的多出约150平方度,由此又可以进一步节省出很多的时间来。最终估算下来可以节约的时间为
18.2天或6300次普通曝光。

\MT{\heiti 目前对“极深场”观测所采取策略是,仅在转动角度很小的时候(小于10度)采取权重的优化;
当上一次观测天区属于“极深场”的时候,会尽可能不进行转动,继续重复观测这个天区,从而节约一些时间,
留给后期观测150秒曝光的区域。}

\RT{\heiti 所有的优化都应该尽可能地在小角度转动的前提下来实现!!!}

\subsection{7月25日}
这两天重点排查了CMG转动角度的计算。最开始的时候觉得是CMG转动角的计算有误(在角度单位没有
搞错的前提下),但是检查后发现其实没有错误。但是原先我采用的转动方式在某些特殊情况下(例如
在靠近黄极的时候)给出的转动角非常大,接近180度。这显然不是很优化的转动方式,这也许是造成
模拟效率变低的一个重要原因。

针对这个问题,重新设计了计算CMG转动角的函数,并且确保了帆板面的指向经转动后依然保持向着太阳。
现在已经做了很多的测试,并且与旧版本的函数进行了对比。在“新旧”指向的经度差不超过90度的时候,两个
函数给出的转动角完全一致,符合预期;而在其他情况下,新函数给出的转动角通常会小于旧函数的结果,
但也不排除在某些情况下,新函数给出的转动角大于旧函数的结果。通过蒙特卡洛模拟,发现新函数给出
的转动角度平均来讲还是要小于旧函数的结果。

现在已经将策略部分回归到张鑫原先的设定,然后重新开始模拟了,期待出现奇迹吧。。。

睡了(凌晨2点58分50秒)。
\input{chapters/internal}
\input{chapters/appendix}



\end{document}
