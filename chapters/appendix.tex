% !TEX root = ../CSS-OS.tex

\begin{appendix}

\chapter{文档编写说明}
附录部分主要记录一些杂七杂八的事情。

本文档是用Latex所编写的。具体使用的Latex程序为2017版的TexLive套装,latex模板为ctexrep。需要注意的是,若
使用较低版本的TexLive,可能会由于版本原因导致文档无法正常编译。另外Latex源文件的编码可能会造成编译中文时出现
乱码,解决办法是将所有源文件都以UTF8的编码来进行保存。

为了区分所写的内容的“确定度”,即这些内容是否经常变更,建议使用不同颜色来标记。暂时规定以“黑色”来编写那些几乎不会变更
的内容;那些暂时性的、需要经常变更的内容则全部用“彩色”。


\chapter{一些有用的参考资料或软件}

\section{Healpix}
Healpix可以用于将天区进行等面积划分。参考网址如下:
\begin{itemize}
\item Healpix 官网:\url{http://healpix.sourceforge.net/documentation.php}
\item Healpix Python 包官网:\url{http://healpy.readthedocs.io/en/latest/tutorial.html}
\end{itemize}

\section{PyEphem}
在进行一些球面坐标系之间的转换的时候发现了这款方便的工具。
网址:\url{http://rhodesmill.org/pyephem/index.html}。

\section{坐标系统和时间系统}
\url{http://netclass.csu.edu.cn/JPKC2007/CSU/02GPSjpkch/jiao-an/2.1.htm}。



\chapter{编程的一些注意事项}
本章内容的目的是为了更有效地设计和编写程序,并使得程序的调试、维护和管理等变得更加容易。主要内容包括:
变量、函数名的命名法,函数的输入、输出(返回值),错误(异常)的处理和程序的终止等等。
\RT{\textbf{该编排模拟程序的主要编程语言为C++语言,部分代码由C语言所编写。}}

\section{代码注释的编写}
强烈建议在写代码的过程中,在适当的、关键的或者是需要进一步改进和优化的地方写下详细的注释,这样不仅方便
代码作者本人日后的维护,同时也方便其他人来协助检查和改进该代码。

\section{变量、函数的命名法}
推荐采用“匈牙利命名法”,具体规则可以直接通过网络搜索得到。当然也有其他一些命名法则,
比如\url{http://blog.csdn.net/f_zyj/article/details/51510085}中所介绍的。具体使用哪种命名法则,等具体
开始重新编写程序前再进行商讨。

\section{函数的输入、输出(返回值)}


\section{错误、异常的处理和程序终止}


\end{appendix}
