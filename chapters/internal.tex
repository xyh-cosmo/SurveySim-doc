% !TEX root = ../CSS-OS.tex

\chapter{内部资料}
\label{chap_internal}

\section{巡天任务规划专题协调}
\label{internal_sec_1}

\subsection{机动角速度}
\begin{itemize}
\item[1] 姿态角速度0.6\textdegree/s为12.5tCZ-7方案的设计目标,历经13.5cTZ-7(0.54\textdegree / s)、
13.5tCZ-5B(0.46\textdegree / s)、15.5tCZ-5B(0.37\textdegree / s),转动惯量增长与角速度能力下降对应;
\item[2] 同时平台姿态机的优化措施是力保20\textdegree(平台按序列统计,占比$65\%$)以下机动稳定用时不变,代价是
大角度机动时间更长。
\end{itemize}
 
\subsection{巡天任务规划协调}
\subsubsection{帆板面积}
$30\%$转换效率为地面实验室$25\textdegree C$条件下太阳电池片的标称值,在轨情况受电池片组合匹配、紫外辐照、
静电泄漏等因素影响,发电能力损失$8\%$。太阳翼在轨工作温度在$100\textdegree$左右,功率损失约$22\%$。

10年寿命考虑空间辐照的影响,性能衰减$7\%$~\footnote{\RT{此处$7\%$的衰减该如何理解?是在$30\%$的转换效率
的基础上衰减,还是说说整体“性能”的衰减(这样在后期时候的效率就是乘以$(1-7\%)$)?Boss詹的理解是在$30\%$的
转换效率的基础上衰减,因此10年后$30\%$变为$23\%$。到底该如何理解,可以结合限制条件中列出的条件进行检验。
}\BT{4月18日上午询问了五院的孙国童,对方的回答是这些系数都是相乘的关系,不是在$30\%$的基础上减去某个数字。}}。

阳照区内,要求帆板发电能力同时满足7500W负载供电需求和电池充电需求,考虑布片系数0.87、\MT{可靠性冗余、蓄电池
充放电损失},$75m^2$帆板的余量较小。

光学舱系统姿态稳定度目前按照0.01的帆板结构阻尼开展方案阶段仿真,正在组织能源功能按$0.02\sim 0.03$的结构
阻尼开展帆板方案优化。

后续一体化设计,平台和光学设施可结合产品继续优化整舱功耗,帆板面积和刚度不会有量级变化。

\subsubsection{巡天效率}
\begin{itemize}
\item[1] 沟通上报中央的具体思路,为了回答工程总体力保35000平方度的问题,平台建议分两个层次回答:
    \begin{itemize}
	\item 第一个层次是基于“时分”时期的科学要求,在星等不变的条件下可保35000;
	\item 第二个层次按提高科学产出用面积换星等。
    \end{itemize}
\item[2] 巡天规划中对效率影响较显著的因素,偏置角范围、机动稳定时间、观测序列优化、主光机杂散光抑制等,
    \begin{itemize}
	\item 是否可以排出优先级次序,方便确定后续工作的侧重点;
	\item 各影响因素的解决建议;
    \end{itemize}
\item[3] 入轨初期帆板发电能力实测优于指标的部分,可以提供给巡天观测,用于扩大偏置\footnote{偏置是什么意思?}
观测范围;平台理解是否正确:巡天任务前期,扩大偏置角对巡天效率的提升不明显,且小角度机动比例下降;
\item[4] 了解影响凝视时间(深度150s、极深度250s)的主要因素,凝视与机动次数1:1,在不影响观测星等的前提下,
优化凝视时间对巡天效率同样重要。另外,仿真计算时,在杂散光较低的天去是否可以调整凝视时间的取值?
\end{itemize}

\subsection{\RT{\heiti 一些补充说明:}}
一共有三组电池,电池的参数为:$90~\rm{aH}$,$100~\rm{V}$;设施的功耗为3500W(最小3300W)。


