% !TEX root = ../CSS-OS.tex

\chapter{模拟程序的优化}

\section{7月}

\RT{\heiti 前阵子发现了程序中的一个严重bug,就是在计算CMG的转动角度的时候,传递的天区位置没有正确转换
到弧度单位下的数值,因而所给出的序列全部都错了!修正了bug之后重新运行了几组,发现不能在10年内完成既定
的巡天任务。}

\subsection{7月18日}

\BT{\heiti 在上个周末,苦苦思索、分析模拟运行的中间状态后,意识到张鑫原先设计的策略中有这样一条,就是
在当前搜索到的可观测天区内,如果有“极深场”,那么就优先观测“极深场”,这导致了很多不必要的大角度转动。
优先观测“极深场”引起的这些不必要的大角度转动造成了很多“可观测时间”的浪费,最终后果就是减少了10年里
真正可以用于曝光的时间。}

\GT{\heiti 由此可以提炼出一条优化规划的法则:“任何时候都要减少转动的时间”!}

从7月17日傍晚6点半至18日半夜,已经完成4组取消了“极深场优先策略”的模拟,虽然离预期的目标还差一些,
但是相比之下时间的利用率有了很大的改进,最终未能被观测到的天区数目减少了很多。尽管已经能保证17500
平方度的大面积巡天,但是“极深场”的面积离目标还差100多平方度。

将“天区覆盖率”随时间的关系统计出来之后,发现从8.5年开始,覆盖速率有较为明显的下降。下降的主要原因
是低黄纬、低银纬的天区很快就被覆盖完了,剩下的未观测区域相互之间离得比较远,因此望远镜的指向在从
一个区域转动到另外一个区域的时候就需要更多的时间,进而造成了时间上的不必要的浪费。

18日晚又提交了4组模拟,这一次对高纬度的区域进行了调整,提高了高纬度的阈值,尽可能使得高纬度的区域
尽快被观测完,延缓低纬度区域被观测完的时间,这样做的目的是为了在10年内能够让“天区覆盖率”维持在一个
平均水平,不至于在后期的时候速率降到很低的值。\RT{这4组模拟所采用的天区划分完全一致,只是对“高纬度”
所选取的阈值不同;另外将“连续观测优先”的策略延迟到第8年才开始。}

\subsection{7月19日}
今天想到,针对昨天发现的“极深场”面积不达标的问题,应该可以比较简单地解决。在根据CMG转动角度进行
权重修改的代码部分,在角度很小的两种情况下(小于5度和10度的两种情况下,此时转动所需的时间较少),
可以加一个判断语句,如果当前“预选”天区属于“极深场”,那么就进一步优化权重,这样一来也可以保证不需要
大角度的转动了。下午3点40分的时候已经提交了4组这样的模拟,明早看结果。

仅仅使用“天区覆盖率”还不能够比较准确地来度量巡天效率,应该选择一段时间内(比如60天)的总曝光时间
所占的比例。目前程序中还不能实时地计算这个数值,但根据所生成的序列还是可以进行统计的。(需要写一
个脚本来进行统计)

\RT{因为“极深场”区域的每一个天区都需要重复观测4次,那么在满足一定条件的时候,就可以省去转动、直接
开始多次曝光了。但是要进行这样的判断需要提前知道轨道的情况,感觉会比较的复杂。。。}

\MT{突然又意识到,其实使用“连续观测”的策略进行权重修改会干扰后面根据CMG转动角大小进行权重分配,
于是干脆将“连续观测”这个策略取消,现在已经再次提交了四组模拟。(张鑫原先实现的“连续观测”策略可能会
导致大角度的转动???)}

\subsection{7月20日}

还有一些小技巧可以进一步减少时间的浪费。一个具体的例子就是在当前比较容易观测的区域属于的时候,可以
连续地进行曝光,(几乎)不需要进行转动,从而可以节省下70秒的时间(根据XXX提供的数据,角度小于
1度的转动需要的时间为70秒)。要使得“极深场”面积达到400平方度,总的曝光次数至少为43636次,可以
利用来节约时间的曝光次数为32727次。如果其中有一半可以用来不转动、节约时间,那么一共可以节省下约
13天的时间,这些时间又几乎都是可以进行正常观测的,因此转化为大面积巡天曝光次数,大约为4580次
(具体估算为400/1.1*30*1.5*70/(150+100)=4581)。然而,目前实际给“极深场”规划的天区面积要多出很多,
比实际需要的多出约150平方度,由此又可以进一步节省出很多的时间来。最终估算下来可以节约的时间为
18.2天或6300次普通曝光。

\MT{\heiti 目前对“极深场”观测所采取策略是,仅在转动角度很小的时候(小于10度)采取权重的优化;
当上一次观测天区属于“极深场”的时候,会尽可能不进行转动,继续重复观测这个天区,从而节约一些时间,
留给后期观测150秒曝光的区域。}

\RT{\heiti 所有的优化都应该尽可能地在小角度转动的前提下来实现!!!}

\subsection{7月25日}
这两天重点排查了CMG转动角度的计算。最开始的时候觉得是CMG转动角的计算有误(在角度单位没有
搞错的前提下),但是检查后发现其实没有错误。但是原先我采用的转动方式在某些特殊情况下(例如
在靠近黄极的时候)给出的转动角非常大,接近180度。这显然不是很优化的转动方式,这也许是造成
模拟效率变低的一个重要原因。

针对这个问题,重新设计了计算CMG转动角的函数,并且确保了帆板面的指向经转动后依然保持向着太阳。
现在已经做了很多的测试,并且与旧版本的函数进行了对比。在“新旧”指向的经度差不超过90度的时候,两个
函数给出的转动角完全一致,符合预期;而在其他情况下,新函数给出的转动角通常会小于旧函数的结果,
但也不排除在某些情况下,新函数给出的转动角大于旧函数的结果。通过蒙特卡洛模拟,发现新函数给出
的转动角度平均来讲还是要小于旧函数的结果。

现在已经将策略部分回归到张鑫原先的设定,然后重新开始模拟了,期待出现奇迹吧。。。

睡了(凌晨2点58分50秒)。