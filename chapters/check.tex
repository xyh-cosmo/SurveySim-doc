% !TEX root = ../CSS-OS.tex

\chapter{模拟结果的检验}

{\heiti 检验的工作与唐怀金合作。}


\section{检验模拟结果的基本思路及各项判断条件的具体定义}


\subsection{卫星受到光照(阳照区)的判断}
\RT{此处内容由唐怀金提供}

模型一:地球中心$O$,与卫星$E$和太阳$S$。$R$是地球半径.太阳角半径16角分,地平线上蒙气差为35.4角分。
卫星受到光照的判断条件是 
\begin{equation}
\theta_0 \le \theta_1+ \theta_2+ \theta_3
\end{equation}
其中
\begin{eqnarray}
\theta_0 &=& \arccos \frac{\bm{OS} \cdot \bm{OE}}{|\bm{OS}| \cdot |\bm{OE}|},\\
\theta_1 &=& \arccos{\frac{R}{|\bm{OS}|}},\\
\theta_2 &=& \arccos{\frac{R}{|\bm{OE}|}},\\
\theta_3 &=& (16+35.4\times 2)/60.
\end{eqnarray}
 
模型二:卫星未受到光照的判断条件是地球中心到太阳与卫星之间的连线的距离$h$满足如下关系
\begin{eqnarray}
h < R \cos\left( \frac{16+35.4\times2}{60} \right),
\end{eqnarray}
且$\theta_0$为钝角,此模型为最早使用的模型。 其中 
\begin{eqnarray}
h=\frac{\bm{OS} \cdot \bm{OE}}{|\bm{SE}|}
\end{eqnarray}

\subsection{杂散光模型(唐怀金提供)}
原点位于地球中心$O$,目标点P的坐标为$(P_x,P_y,P_z)$,卫星的位置STL为$(STL_x,STL_y,STL_z)$,
太阳位置SUN为$(SUN_x,SUN_y,SUN_z)$,地球半径为$R$。天顶距$ANGLE_z$,暗边角阈值$ANGLE_{dark}$,
亮边角阈值$ANGLE_{light}$.

遮挡角:
\begin{eqnarray}
ANGLE_1 = \arcsin\left(\frac{R}{L_{OP}}\right),
\end{eqnarray}
简便起见,这里不严格区分弧度与角度的差别。

卫星在地球上的视线切线是否与晨昏线相交分为三种情况:
\begin{itemize}

\item [一:]与晨昏线不相交且卫星位于阴影区域,计算观测目标点的天顶距,判断是否为暗边角的判断条件:
\begin{eqnarray}
ANGLE_z \le 180-ANGLE_1-ANGLE_{dark}
\end{eqnarray}

\item [二:]与晨昏线不相交且卫星位于阳照区,计算观测目标点的天顶距,判断是否为亮边角的判断条件:
\begin{eqnarray}
ANGLE_z \le 180-ANGLE_1-ANGLE_{light}
\end{eqnarray}

\item [三:]与晨昏线相交。\\
第一步:以亮边角的天顶距做初始判断,满足条件则通过:
\begin{eqnarray}
ANGLE_z \le 180-ANGLE_1-ANGLE_{light};
\end{eqnarray}

第二步:不满足第一步条件,但是满足以下两个条件:
\begin{eqnarray}
ANGLE_z &\ge& 180-ANGLE_1-ANGLE_{light}\\
ANGLE_z &\le& 180-ANGLE_1-ANGLE_{dark}
\end{eqnarray}
则计算卫星、地心和目标点共面的切点$P_{Q}$(可由平面几何推导),切点必须位于阴影区。
若能计算出晨昏线与视线切线环的焦点,则考虑阳照区的影响。

\end{itemize}

\section{目前的检验结果}
\begin{itemize}
\item 阳照区\\
目前对各项限制条件的检验结果中“阳照区”的判断有$4.5\%$的不符合率,是目前所有限制条件中
最严重的情况。目前已经将这些不符合的模拟结果调取出来进行细致的检查。\RT{目前猜测,出现
这么高的不符合率应该是由张鑫的程序与唐怀金的检验程序之间的一些假设之间的差异所导致的:
张鑫的程序对阳照区的判断基于“照射到地球的太阳光为平行光”这样一个假设,相比较之下,唐怀金
的检验程序在这一块的处理更加接近实际情况。}

\item SAA区域\\

\item 太阳能帆板\\


\item 地球杂散光、地气光\\


%\item 暗边、亮边夹角\\


\item 太阳夹角\\


\item 月球夹角\\

\end{itemize}
\section{检验中出现的不符合情况及原因排查}


\section{检验各项条件的具体思路}
\RT{\heiti 这个是一个全新的章节,目的是详细讨论检验各种限制的具体做法,具体如何实现,甚至
包括如何检验这些测试本身(这就需要构造出一些具体的、知道结果的例子来来进行验证)。这一
章节重点围绕动态限制条件,也就是CMG和能源这两方面;其余的条件都是静态的,或者准确点说是
前后没有什么因果联系的,比如各种遮挡角度的计算。}

\subsection{CMG}
昨天(Wed Jul-10, 2018)在生成相机运行序列的时候,顺便检验了一下连续两次曝光之间,由于
机动原因导致的指向随时间的变化。

检验的具体思路如下:知道了两个天区的指向,同时也知道有一个旋转矩阵,可以描述这个转动过程,
并保证不产生像旋(这个矩阵的公式可以方便地推导出来)。假设这两个天区的指向分别为
$\bm{P}_{init}$和$\bm{P}_{final}$,相应的旋转矩阵记为
$\bm{R}(t,t_{init};\bm{P}_{init},\bm{P}_{final})$,那么旋转过程的任意时刻的指向可以写作:
\begin{eqnarray}
\bm{P}(t) = \bm{R}(t,t_{init}) \cdot \bm{P}_{init} \text{,}
\end{eqnarray}
其中$t_{init} \le t \le t_{final}$;此处为了简便,旋转矩阵中忽略掉初始和末尾指向。

整个拍摄周期划分为多个“状态”,每一个“状态”又是由各个“组部件的状态”集合而成的;这些组部件的工作
状态在当前的“状态”下维持不变。具体检验过程中我是这么来做的:在转动到下一个指向之前的那个状态
下,根据巡天编排模拟程序输出结果中的前后两个指向来确定出旋转轴,然后假设旋转是匀速的,那样
一来就可以直接计算出旋转过程中任意时刻的指向了。当计算出末尾指向后与模拟结果给出的指向进行
比较,发现存在很大的差别。

为了找出原因,仔细检查了旋转矩阵的计算等等,确认无误。然后就是检查巡天编排的模拟程序,找到
计算CMG旋转角度的那些代码,然后一眼就看到在将天区指向传递给计算旋转角度的函数时,忘记做
角度的单位转换了。当时还根据天文习惯和传统数学对球坐标系的不同使用习惯,将天顶角($\theta$)
进行了变换,但就是忘记把黄经(ra)和黄纬(dec)的单位给转换到弧度了。\RT{(重大失误!
这样一来,不仅CMG的转动角以及CMG的状态受到影响,连转动时间也受到影响了)}

\subsection{Solar Energy}
